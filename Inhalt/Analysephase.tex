% !TEX root = ../Projektdokumentation.tex

\section{Analysephase}
\label{sec:Analysephase}

\subsection{Ist-Analyse}
\label{sec:IstAnalyse}

Der bisherige \hyperlink{Onboarding}{\textcolor{AOBlau}{Onboarding}}-Prozess stellt neue Mitarbeiter vor die Herausforderung, dass zentrale Informationen über interne Prozesse, wie die Standorte von Dokumentationen, eingesetzte Technologien und wichtige Einstiegshilfen, nicht rechtzeitig zur Verfügung stehen. Obwohl wir eine Entwicklerplattform anbieten, auf der alle notwendigen Dokumentationen gesammelt sind, entdecken neue Entwickler diese häufig erst spät. Zudem sind neue Mitarbeiter oft nicht in die relevanten Microsoft Teams-Kanäle integriert, was die Kommunikation und den Zugriff auf wichtige Informationen zusätzlich verzögert.

Da jeder neue Entwickler früher oder später in unserem \hyperlink{GitLab}{\textcolor{AOBlau}{GitLab}} System arbeitet, besteht hier die Chance, den \hyperlink{Onboarding}{\textcolor{AOBlau}{Onboarding}}-Prozess zu automatisieren. Momentan fehlen jedoch automatisierte Prozesse, die neue Entwickler nahtlos in die vorhandenen Systeme einbinden.

Um dieses Problem zu lösen, haben wir beschlossen, eine \hyperlink{GitLabSystemhooks}{\textcolor{AOBlau}{GitLab-Systemhook}} zu implementieren, die auf \texttt{user\_create} Events lauscht und entsprechend reagiert. Zum Zeitpunkt des Projektbeginns waren bereits drei separate \hyperlink{GitLabSystemhooks}{\textcolor{AOBlau}{Systemhooks}} im Einsatz. Daher entschieden wir uns für die Entwicklung einer zentralen \hyperlink{GitLabSystemhooks}{\textcolor{AOBlau}{Systemhook}}, die alle \hyperlink{GitLabEvent}{\textcolor{AOBlau}{GitLab Events}} an ein zentrales \hyperlink{SNS}{\textcolor{AOBlau}{AWS SNS Topic}} (Simple Notification Service) sendet. Von dort aus können verschiedene Anwendungen durch \hyperlink{SQS}{\textcolor{AOBlau}{SQS Queues}} (Simple Queue Service) und spezifische Filterregeln gezielt auf die \hyperlink{GitLabEvent}{\textcolor{AOBlau}{GitLab Events}} reagieren. Diese Lösung bietet einen zentralen Punkt für \hyperlink{GitLabEvent}{\textcolor{AOBlau}{GitLab Events}} und ermöglicht es uns, den \hyperlink{Onboarding}{\textcolor{AOBlau}{Onboarding}}-Prozess effizient zu erweitern und zu automatisieren.


\subsection{Wirtschaftlichkeitsanalyse}
\label{sec:Wirtschaftlichkeitsanalyse}

Das Projekt zur Automatisierung des \hyperlink{Onboarding}{\textcolor{AOBlau}{Onboardings}} ist für die TUI Group von großem wirtschaftlichen Nutzen. Die Automatisierung sorgt dafür, dass neue Entwickler sofort Zugriff auf alle relevanten Ressourcen erhalten, ohne dass manuelle Eingriffe notwendig sind. Dadurch werden sowohl Verzögerungen als auch die Gefahr, Mitarbeiter zu übersehen oder Zugänge zu vergessen, deutlich reduziert. Dies steigert nicht nur die Effizienz der neuen Entwickler, sondern entlastet auch das IT-Team, das sich auf wichtigere Aufgaben konzentrieren kann. Durch die Einsparung von Zeit und die Reduktion der Arbeitsunterbrechungen werden langfristig Kosten gesenkt und die Produktivität gesteigert.

\subsubsection{\gqq{Make or Buy}-Entscheidung}
\label{sec:MakeOrBuyEntscheidung}

Es existiert keine fertige Softwarelösung, die die spezifischen Anforderungen der TUI Group vollständig abdeckt. Besonders die Integration mit den bereits genutzten \hyperlink{GitLabSystemHooks}{\textcolor{AOBlau}{GitLab-Systemhooks}} und die spezifischen Anpassungen, die für die internen Abläufe notwendig sind, erfordern eine stark angepasste, maßgeschneiderte Lösung. Aus diesem Grund wurde entschieden, das Projekt intern umzusetzen, um eine optimale Integration in die bestehende Infrastruktur sowie eine hohe Flexibilität für zukünftige Anpassungen sicherzustellen.

\subsubsection{Projektkosten}
\label{sec:Projektkosten}

Die Kosten für die Durchführung des Projekts setzen sich aus Personal- und Ressourcenkosten zusammen. Bei der Berechnung des Stundensatzes sowohl für Auszubildende als auch für Senior Entwickler werden Arbeitgeberanteile und Sozialversicherungen berücksichtigt, um den tatsächlichen Stundensatz realistisch darzustellen.
\newline
Ein Auszubildender im dritten Lehrjahr verdient laut Tarifvertrag monatlich 1486 € brutto, was einem Stundenlohn von etwa 11,68 € entspricht:

\begin{align}
    \frac{(1486 \, \text{€} \cdot 1.21)}{20 \, \text{Tage/Monat} \cdot 7.7 \, \text{Stunden/Tag}} &= \frac{1.798.06 \, \text{€/Monat}}{154 \, \text{Stunden/Monat}} \tag{1} \\
    &\approx 11,68 \, \text{€/Stunde} \tag{2}
\end{align}


Für die Mitarbeit eines Senior Entwicklers, der für Fachgespräche konsultiert wurde, wird ein monatliches Gehalt von 6069 € zugrunde gelegt, was einem Stundenlohn von etwa 45,86 € ergibt:

\begin{align}
    \frac{(6069 \, \text{€} \cdot 1,21)}{20 \, \text{Tage/Monat} \cdot 7.7 \, \text{Stunden/Tag}} &= \frac{7.343,49 \, \text{€/Monat}}{154 \, \text{Stunden/Monat}} \tag{1} \\
    &\approx 47,69 \, \text{€/Stunde} \tag{2}
\end{align}

Zusätzlich zu den Personalkosten werden pauschale Ressourcenkosten von 15 € pro Stunde berücksichtigt, um die Nutzung von Büroarbeitsplätzen, Hardware, Software und Strom abzudecken. Der Stundensatz für einen Auszubildenden beläuft sich auf 10,96 €, während der eines Senior Entwicklers 45,86 € beträgt. Für die Ressourcennutzung wird ein einheitlicher Stundensatz von 15 € angesetzt. Die nachfolgende Tabelle zeigt eine detaillierte Übersicht der Projektkosten für die einzelnen Arbeitsschritte.
\begin{table}[h!]
    \centering
    \footnotesize % Kleinere Schriftgröße
    \renewcommand{\arraystretch}{1.2} % Zeilenhöhe anpassen
    \setlength{\tabcolsep}{7pt} % Spaltenabstand reduzieren
    \begin{tabular*}{\textwidth}{|p{2.7cm}|p{3.0cm}|p{1.5cm}|r|r|p{1.9cm}|} % Spaltenbreiten anpassen
    \hline
    \textbf{Vorgang}       & \textbf{Mitarbeiter}                 & \textbf{Zeit (h)} & \textbf{Personal (€)\footnotemark[6]} & \textbf{Ressourcen (€)\footnotemark[7]} & \textbf{Gesamt (€)} \\ \hline
    Entwicklungskosten     & 1 x Auszubildender                  & 80                & 934,4                 & 1.200                 & 2.134,4           \\ \hline
    Fachgespräch           & 2 x Senior Entwickler\newline 1 x Auszubildender & 3                 & 321,18                 & 135,00                   & 456,18             \\ \hline
    Code-Review            & 1 x Senior Entwickler                      & 4                 & 190,76                 & 60,00                    & 250,76             \\ \hline
    Abnahme                & 2 x Senior Entwickler                      & 1                 & 95,38                  & 30,00                    & 125,38              \\ \hline
    \multicolumn{5}{|r|}{\textbf{Projektkosten gesamt}} & \textbf{2.966,72} \\ \hline
    \end{tabular*}
    \caption{Kostenaufstellung}
    \label{tab:Kostenaufstellung}
\end{table}

\footnotetext[6]{Personalkosten pro Vorgang = Anzahl Mitarbeiter $\cdot$ Zeit $\cdot$ Stundensatz.}
\footnotetext[7]{Ressourcenbeitrag pro Vorgang = Anzahl Mitarbeiter $\cdot$ Zeit $\cdot$ 15 € (Ressourcenbeitrag pro Stunde).}

\subsubsection{Amortisationsdauer}
\label{sec:Amortisationsdauer}

Im folgenden Abschnitt soll ermittelt werden, ab welchem Zeitpunkt sich die Entwicklung der Anwendung amortisiert hat. Dieser Wert ermöglicht es, die wirtschaftliche Sinnhaftigkeit des Projekts zu bewerten und zu erkennen, ob sich durch die Umsetzung langfristige Kostenvorteile ergeben. Die Amortisationsdauer wird berechnet, indem die Anschaffungskosten durch die laufende Kostenersparnis geteilt werden, die das neue Produkt erzielt. 

Durch die Automatisierung des Übertragungsprozesses ließe sich erhebliche Bearbeitungszeit einsparen, was zu einer Reduzierung der Personalkosten führt.

Im \Anhang{app:User} ist dargestellt, dass wir monatlich etwa 150 neue Benutzer hinzugewinnen. Für jeden neuen Benutzer werden etwa 5 Minuten benötigt, um ihnen die relevanten Informationen zu internen Prozessen und Dokumentationsstandorten bereitzustellen und sie in die Teamgruppe hinzuzufügen. Durch den Automatisierungsprozess kann dieser Zeitaufwand vollständig eingespart werden.

Multipliziert man die monatlichen neuen Benutzer (150) mit den 5 Minuten pro Benutzer, ergibt das jährlich:

\[
150 \, \text{Benutzer} \times 5 \, \text{Minuten} \times 12 \, \text{Monate} = 9000 \, \text{Minuten/Jahr} \quad \text{oder} \quad 150 \, \text{Stunden/Jahr}
\]

Da der Onboarding-Prozess meist von Senior-Entwicklern durchgeführt wird, wird für die Berechnung der Einsparungen der Stundensatz eines Senior Entwicklers von \eur{47,69} verwendet. Das führt zu einer jährlichen Einsparung von:

\[
\text{Einsparung} = 150 \, \text{Stunden} \times \eur{47,69} \approx \eur{7.153,5}
\]

Die Amortisationsdauer des Projekts kann berechnet werden als:

\[
\text{Amortisationsdauer (in Jahren)} = \frac{\text{Anschaffungskosten}}{\text{jährliche Einsparung}} = \frac{2.966,72 \, \text{€}}{7.153,5 \, \text{€}} \approx 0,415 \, \text{Jahre}
\]

Um die Amortisationsdauer in Tagen zu erhalten, multiplizieren wir mit 365:

\[
\text{Amortisationsdauer (in Tagen)} \approx 0,415 \, \text{Jahre} \times 365 \, \text{Tage/Jahr} \approx 151 \, \text{Tage}
\]

Die Amortisationsdauer des Projekts beträgt somit etwa 151 Tage, was die wirtschaftliche Effizienz des Projekts verdeutlicht. Eine graphische Darstellung ist im Anhang \ref{app:amo} zu finden.

\subsection{Nicht-monetäre Vorteile}
\label{sec:Nutzwertanalyse}

Die Wirtschaftlichkeitsanalyse hat bereits eine ausreichende Grundlage für die Realisierung des Projekts geliefert, sodass auf eine detaillierte Analyse nicht-monetärer Vorteile verzichtet wird. Allerdings sind die qualitativen Verbesserungen, die das Projekt mit sich bringt, von großer Bedeutung.

Diese unmittelbare Verfügbarkeit von Informationen führt dazu, dass neue Teammitglieder schneller produktiv werden können und sich auf ihre Kernaufgaben konzentrieren, anstatt Zeit mit administrativen Tätigkeiten zu verlieren. Zudem wird die Belastung der bestehenden Entwicklerteams deutlich verringert, da weniger Zeit für die Einrichtung und Verwaltung von Benutzerzugängen aufgewendet werden muss. Diese Effekte tragen nicht nur zu einer besseren Zusammenarbeit innerhalb der Teams bei, sondern fördern auch eine angenehme Arbeitsatmosphäre, in der sich alle Teammitglieder auf ihre wichtigen Aufgaben konzentrieren können.
