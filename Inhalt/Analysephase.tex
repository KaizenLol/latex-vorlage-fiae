% !TEX root = ../Projektdokumentation.tex

\section{Analysephase}
\label{sec:Analysephase}

\subsection{Ist-Analyse}
\label{sec:IstAnalyse}

Der bisherige Onboarding-Prozess stellt neue Mitarbeiter vor die Herausforderung, dass zentrale Informationen über interne Prozesse, wie die Standorte von Dokumentationen, eingesetzte Technologien und wichtige Einstiegshilfen, nicht rechtzeitig zur Verfügung stehen. Obwohl wir eine Entwicklerplattform anbieten, auf der alle notwendigen Dokumentationen gesammelt sind, entdecken neue Entwickler diese häufig erst spät. Zudem sind neue Mitarbeiter oft nicht in die relevanten Microsoft Teams-Kanäle integriert, was die Kommunikation und den Zugriff auf wichtige Informationen zusätzlich verzögert.

Da jeder neue Entwickler früher oder später in unserem GitLab-System arbeitet, besteht hier die Chance, den Onboarding-Prozess zu automatisieren. Momentan fehlen jedoch automatisierte Prozesse, die neue Entwickler nahtlos in die vorhandenen Systeme einbinden.

Um dieses Problem zu lösen, haben wir beschlossen, eine GitLab Systemhook zu implementieren, die auf \texttt{user\_create}-Ereignisse lauscht und entsprechend reagiert. Zum Zeitpunkt des Projektbeginns waren bereits drei separate Systemhooks im Einsatz. Daher entschieden wir uns für die Entwicklung einer zentralen Systemhook, die alle GitLab-Ereignisse an ein zentrales \textbf{AWS SNS} Topic (Simple Notification Service) sendet. Von dort aus können verschiedene Anwendungen durch \textbf{SQS} Queues (Simple Queue Service) und spezifische Filterregeln gezielt auf die GitLab-Ereignisse reagieren. Diese Lösung bietet einen zentralen Punkt für GitLab-Ereignisse und ermöglicht es uns, den Onboarding-Prozess effizient zu erweitern und zu automatisieren.

\subsection{Wirtschaftlichkeitsanalyse}
\label{sec:Wirtschaftlichkeitsanalyse}

Das Projekt zur Automatisierung des Onboardings ist für die \ac{TUI} InfoTec GmbH von großem wirtschaftlichen Nutzen. Die Automatisierung sorgt dafür, dass neue Entwickler sofort Zugriff auf alle relevanten Ressourcen erhalten, ohne dass manuelle Eingriffe notwendig sind. Dadurch werden sowohl Verzögerungen als auch die Gefahr, Mitarbeiter zu übersehen oder Zugänge zu vergessen, deutlich reduziert. Dies steigert nicht nur die Effizienz der neuen Entwickler, sondern entlastet auch das IT-Team, das sich auf wichtigere Aufgaben konzentrieren kann. Durch die Einsparung von Zeit und die Reduktion der Arbeitsunterbrechungen werden langfristig Kosten gesenkt und die Produktivität gesteigert.

\subsubsection{\gqq{Make or Buy}-Entscheidung}
\label{sec:MakeOrBuyEntscheidung}

Es existieren zwar bereits Softwarelösungen, die Teile des Onboardings automatisieren könnten, jedoch gibt es kein Produkt, das die spezifischen Anforderungen der \ac{TUI} InfoTec GmbH vollständig abdeckt. Besonders die Integration mit den bereits genutzten \hyperlink{GitLabSystemHooks}{\textcolor{blue}{GitLab-SystemHooks}} und die spezifischen Anpassungen, die für die internen Abläufe nötig sind, erfordern eine maßgeschneiderte Lösung. Aus diesem Grund wurde beschlossen, das Projekt intern umzusetzen, um eine optimale Integration in die bestehende Infrastruktur und eine hohe Flexibilität für zukünftige Anpassungen sicherzustellen.

\subsubsection{Projektkosten}
\label{sec:Projektkosten}

Die Kosten für die Durchführung des Projekts setzen sich sowohl aus
Personal-, als auch aus Ressourcenkosten zusammen. Laut Tarifvertrag verdient ein Auszubildender
im dritten Lehrjahr pro Monat 1450 € Brutto.

\begin{align}
    8 \, \text{h/Tag} \cdot 220 \, \text{Tage/Jahr} &= 1760 \, \text{h/Jahr} \tag{1}\\
    1450 \, \text{€/Monat} \cdot 13.3 \, \text{Monate/Jahr} &= 19285 \, \text{€/Jahr} \tag{2}\\
    \frac{19285 \, \text{€/Jahr}}{1760 \, \text{h/Jahr}} &\approx 10.96 \, \text{€/h} \tag{3}
\end{align}

Die Kosten, die während der Entwicklung des Projekts anfallen, werden im Folgenden kalkuliert. Die f
-ür die Realisierung des Projekts benötigten Personal- und Ressourcenkosten sind von der Personalabteilung festgelegte Pauschalsätze, die nicht weiter angepasst werden dürfen. Der Stundensatz eines Auszubildenden beträgt demzufolge 10.96€, der eines Mitarbeiters 25€. Die Ressourcennutzung umfasst einen Büroarbeitsplatz, die Hardware- und Softwarenutzung sowie Stromkosten. Hierfür wird von der Personalabteilung ein pauschaler Stundensatz von 15€ vorgegeben. Die folgende Tabelle zeigt die detaillierten Projektkosten auf, die für die einzelnen Vorgänge anfallen.

\begin{table}[h!]
    \centering
    \footnotesize % Kleinere Schriftgröße
    \renewcommand{\arraystretch}{1.2} % Zeilenhöhe anpassen
    \setlength{\tabcolsep}{7pt} % Spaltenabstand reduzieren
    \begin{tabular*}{\textwidth}{|p{2.7cm}|p{3.0cm}|p{1.5cm}|r|r|p{1.9cm}|} % Spaltenbreiten anpassen
    \hline
    \textbf{Vorgang}       & \textbf{Mitarbeiter}                 & \textbf{Zeit (h)} & \textbf{Personal (€)\footnotemark[6]} & \textbf{Ressourcen (€)\footnotemark[7]} & \textbf{Gesamt (€)} \\ \hline
    Entwicklungskosten     & 1 x Auszubildender                  & 80                & 800,00                 & 1.050,00                 & 1.750,00           \\ \hline
    Fachgespräch           & 2 x Mitarbeiter, \newline 1 x Auszubildender & 3                 & 180,00                 & 135,00                   & 315,00             \\ \hline
    Code-Review            & 1 x Mitarbeiter                      & 4                 & 100,00                 & 60,00                    & 160,00             \\ \hline
    Abnahme                & 2 x Mitarbeiter                      & 1                 & 50,00                  & 30,00                    & 80,00              \\ \hline
    \multicolumn{5}{|r|}{\textbf{Projektkosten gesamt}} & \textbf{2.305,00} \\ \hline
    \end{tabular*}
    \caption{Kostenaufstellung}
    \label{tab:Kostenaufstellung}
\end{table}

\footnotetext[6]{Personalkosten pro Vorgang = Anzahl Mitarbeiter $\cdot$ Zeit $\cdot$ Stundensatz.}
\footnotetext[7]{Ressourcenbeitrag pro Vorgang = Anzahl Mitarbeiter $\cdot$ Zeit $\cdot$ 15 € (Ressourcenbeitrag pro Stunde).}

\subsubsection{Amortisationsdauer}
\label{sec:Amortisationsdauer}

Das Projekt bringt erhebliche Zeiteinsparungen mit sich, insbesondere für das IT-Team und die neuen Entwickler. Bei einer durchschnittlichen Zeiteinsparung von 10 Minuten pro Tag und 220 Arbeitstagen im Jahr für jeden der 25 Anwender ergibt sich eine jährliche Zeiteinsparung von etwa 917 Stunden. Bei einem kombinierten Stundensatz von \eur{40} (inklusive Mitarbeiter- und Ressourcenkosten) ergibt sich eine jährliche Einsparung von \eur{36.680}. Die Amortisationsdauer des Projekts beträgt somit etwa 4 Wochen, was das Projekt sehr kosteneffizient macht.

\subsection{Nutzwertanalyse}
\label{sec:Nutzwertanalyse}

Neben den monetären Vorteilen bietet das Projekt auch erhebliche nicht-monetäre Vorteile. Die Automatisierung des Onboardings sorgt dafür, dass neue Entwickler sofort Zugriff auf alle relevanten Ressourcen haben, ohne dass Mitarbeiter ständig an das Einrichten von Zugängen erinnert werden müssen. Dies reduziert Ablenkungen und Unterbrechungen im täglichen Arbeitsablauf, was zu einer höheren Produktivität und einer angenehmeren Arbeitsatmosphäre führt. Dadurch wird nicht nur der Einstieg für neue Entwickler erleichtert, sondern auch die Belastung der bestehenden Teams deutlich verringert.
