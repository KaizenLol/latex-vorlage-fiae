% !TEX root = ../Projektdokumentation.tex
\section{Dokumentation}
\label{sec:Dokumentation}

Die \textbf{Projektdokumentation} beschreibt den gesamten Verlauf der Projektumsetzung, einschließlich der Planung, Durchführung und Tests. Sie dient als Referenz für alle Projektbeteiligten und dokumentiert die wichtigsten Phasen und Entscheidungen im Projekt. IT-spezifische Fachbegriffe werden dabei, soweit möglich, vermieden, um die Verständlichkeit für alle Beteiligten zu gewährleisten.

Die \textbf{Entwicklerdokumentation} enthält die technischen Anforderungen, Konfigurationsparameter und Endpunkte der Anwendung. Sie bietet detaillierte Informationen zu \hyperlink{Umgebungsvariablen}{\textcolor{AOBlau}{Umgebungsvariablen}}, \hyperlink{Flags}{\textcolor{AOBlau}{Flags}} und \hyperlink{Endpoints}{\textcolor{AOBlau}{Endpunkte}} für die Einrichtung und das Deployment der Anwendung. Ein Auszug dieser Dokumentation ist im \Anhang{app:Doc} zu finden.

Die Dokumentationen sind so formuliert, dass sie den Anforderungen der jeweiligen Verwendung gerecht werden. Die Projektdokumentation ist allgemein verständlich formuliert, während die Entwicklerdokumentation technische Details und Anweisungen für die Konfiguration und den Betrieb der Anwendung enthält.
