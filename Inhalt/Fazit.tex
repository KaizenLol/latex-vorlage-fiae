\section{Fazit} 
\label{sec:Fazit}

\subsection{Soll-/Ist-Vergleich}
\label{sec:SollIstVergleich}

Bei einer rückblickenden Betrachtung des Projekts lässt sich festhalten, dass die festgelegten Anforder-
ungen weitgehend erreicht wurden. Alle wesentlichen Funktionen und Anpassungen wurden gemäß den definierten Zielen und Anforderungen umgesetzt.

Die Projektplanung konnte insgesamt eingehalten werden, obwohl kleinere Abweichungen in einzelnen Phasen auftraten. So war in der Dokumentationsphase ein Mehraufwand erforderlich, während die Test- und Kontrollphase weniger Zeit in Anspruch nahm. Diese Abweichungen konnten innerhalb des geplanten Zeitrahmens ausgeglichen werden, sodass das Projekt planmäßig abgeschlossen werden konnte. Die folgende Tabelle zeigt den Soll-/Ist-Vergleich der Projektphasen:

\begin{table}[h]
\centering
\caption{Soll-/Ist-Vergleich}
\label{tab:Vergleich}
\begin{tabular}{|l|c|c|c|}
\hline
\textbf{Projektphase} & \textbf{Geplante Zeit} & \textbf{Tatsächliche Zeit} & \textbf{Differenz} \\
\hline
Analysephase & 6 h & 6 h & 0 h \\
\hline
Entwurfsphase & 6 h & 6 h & 0 h \\
\hline
Implementierungsphase & 34 h & 34 h & 0 h \\
\hline
Test- und Kontrollphase & 18 h & 16 h & -2 h \\
\hline
Dokumentation + Nachbearbeitung & 6 h & 8 h & +2 h \\
\hline
Erstellen der Projektdokumentation und Präsentation & 8 h & 10 h & +2 h \\
\hline
Pufferzeit & 2 h & 0 h & -2 h \\
\hline
\textbf{Gesamt} & \textbf{80 h} & \textbf{80 h} & \textbf{0 h} \\
\hline
\end{tabular}
\end{table}

Der Auftraggeber zeigte sich mit dem Projektergebnis zufrieden, da alle zentralen Anforderungen erfüllt und interne Abläufe optimiert wurden.

\subsection{Lessons Learned}
\label{sec:LessonsLearned}

Im Verlauf des Projekts konnten wertvolle Erfahrungen gesammelt werden, insbesondere hinsichtlich der effizienten Planung und Einhaltung von Projektphasen in einem agilen Umfeld. Die kontinuierliche Rücksprache mit dem Fachbereich erwies sich als entscheidend für die erfolgreiche Umsetzung der Projektanforderungen. Auch die Anwendung der \hyperlink{agil}{\textcolor{AOBlau}{agilen}} Entwicklungsweise und die regelmäßige Einbezi-
ehung der Endnutzer sorgten für eine zielgerichtete und bedarfsorientierte Entwicklung.

\subsection{Ausblick}
\label{sec:Ausblick}

Obwohl alle definierten Anforderungen im Projekt umgesetzt wurden, gibt es Potenzial für zukünftige Erweiterungen. Eine mögliche Weiterentwicklung wäre die Ergänzung zusätzlicher Schnittstellen, um eine automatisierte Anbindung weiterer interner Systeme zu ermöglichen. Auch die Weiterentwicklung der bestehenden Funktionen für eine noch tiefere Integration in die vorhandene Infrastruktur wäre sinnvoll.

Aufgrund des modularen Aufbaus der Anwendung, wie im Abschnitt \ref{sec:Architekturdesign} beschrieben, können diese Anpassungen und Erweiterungen problemlos vorgenommen werden. Der flexible Aufbau der Anwendung gewährleistet somit eine gute Wartbarkeit und Erweiterbarkeit für zukünftige Anforderungen.
