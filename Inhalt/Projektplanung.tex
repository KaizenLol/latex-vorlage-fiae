% !TEX root = ../Projektdokumentation.tex

\section{Projektplanung} 
\label{sec:Projektplanung}

\subsection{Projektphasen}
\label{sec:Projektphasen}

Für die Durchführung des Projekts standen insgesamt 80 Stunden zur Verfügung. Diese Stunden wurden vor Projektbeginn auf verschiedene Phasen der Softwareentwicklung verteilt. Eine Übersicht der groben Zeitplanung und der Hauptphasen ist in
\hyperref[tab:Zeitplanung]{\textcolor{AOBlau}{Tabelle 1}} zu finden. Darüber hinaus können die einzelnen Hauptphasen in kleinere Unterphasen unterteilt werden. Eine detaillierte Darstellung dieser Phasen ist im \Anhang{app:Zeitplanung} zu finden.

\paragraph{}
\ref{tab:Zeitplanung}
\tabelle{Grobe Zeitplanung}{tab:Zeitplanung}{ZeitplanungKurz}\\
Eine detailliertere Zeitplanung findet sich im \Anhang{app:Zeitplanung}.

\subsection{Ressourcenplanung}
\label{sec:Ressourcenplanung}

Die Planung der benötigten Ressourcen ist ein wesentlicher Bestandteil der Projektorganisation. Dazu gehören sowohl Hard- und Software als auch die erforderlichen Räumlichkeiten. Eine detaillierte Übersicht über die verwendeten Ressourcen finden Sie im \Anhang{app:Ressourcen}. 

\subsection{Entwicklungsprozess}
\label{sec:Entwicklungsprozess}

Für die Durchführung des Projekts wurde ein \hyperlink{agil}{\textcolor{AOBlau}{agiler}} Entwicklungsprozess gewählt. Dieser Ansatz ermöglichte es, flexibel auf sich ändernde Anforderungen und Herausforderungen während der Entwickl-
ung zu reagieren.\footnote{Agile Alliance - Agile 101, \cite{agile}.} Der Arbeitsprozess war stark \hyperlink{iterativ}{\textcolor{AOBlau}{iterativ}} geprägt, mit regelmäßigen Rücksprachen und enger Zusammenarbeit mit den Senior-Entwicklern der Abteilung.

Der \hyperlink{iterativ}{\textcolor{AOBlau}{iterative}} Zyklus bestand darin, Arbeitsschritte zu planen, umzusetzen, zu evaluieren und anhand des erhaltenen Feedbacks kontinuierlich zu verbessern. Diese kurzen Iterationszyklen erlaubten es, Ergebnisse frühzeitig zu präsentieren und Rückmeldungen direkt in die nächste Phase einfließen zu lassen. 

Die \hyperlink{agil}{\textcolor{AOBlau}{agile}} Vorgehensweise unterstützte zudem schnelle Anpassungen an neue Anforderungen, die in den regelmäßigen Besprechungen sowie durch Feedback der Senior-Entwickler eingebracht wurden. Durch den Einsatz von Tools wie \hyperlink{MicrosoftTeams}{\textcolor{AOBlau}{Microsoft Teams}} und \hyperlink{Jira}{\textcolor{AOBlau}{Jira}} wurde die Zusammenarbeit und das Projektmanagement gefördert. Diese Tools sorgten dafür, dass der Fortschritt transparent blieb und die Arbeitsschritte effizient geplant und dokumentiert werden konnten.
