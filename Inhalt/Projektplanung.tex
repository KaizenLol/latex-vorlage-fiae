% !TEX root = ../Projektdokumentation.tex

\section{Projektplanung} 
\label{sec:Projektplanung}

\subsection{Projektphasen}
\label{sec:Projektphasen}

Für die Durchführung des Projekts standen insgesamt 80 Stunden zur Verfügung. Diese Stunden wurden vor Projektbeginn
auf verschiedene Phasen der Softwareentwicklung verteilt. Eine Übersicht der groben Zeitplanung und der Hauptphasen ist in
\hyperref[tab:Zeitplanung]{\textcolor{blue}{Tabelle 1}} zu finden. Darüber hinaus können die einzelnen Hauptphasen in kleinere Unterphasen unterteilt werden.
Eine detaillierte Darstellung dieser Phasen ist im \hyperref[app:Zeitplanung]{\textcolor{blue}{Anhang A.1}} auf Seite \textcolor{blue}{\pageref{app:Zeitplanung}}
zu finden.

\paragraph{}
\ref{tab:Zeitplanung}
\tabelle{Grobe Zeitplanung}{tab:Zeitplanung}{ZeitplanungKurz}\\
Eine detailliertere Zeitplanung findet sich im \hyperref[app:Zeitplanung]{\textcolor{blue}{Anhang A.1}}.

\subsection{Ressourcenplanung}
\label{sec:Ressourcenplanung}

Die Planung der benötigten Ressourcen ist ein wesentlicher Bestandteil der Projektorganisation. Dazu gehören sowohl Hard- und Software als auch die erforderlichen Räumlichkeiten. Eine detaillierte Übersicht über die verwendeten Ressourcen finden Sie im \hyperref[app:Ressourcen]{\textcolor{blue}{Anhang A.2}}. 

\subsection{Entwicklungsprozess}
\label{sec:Entwicklungsprozess}

Für die Durchführung des Projekts wurde ein \hyperlink{agil}{\textcolor{blue}{agiler}} Entwicklungsprozess gewählt. Dieser Ansatz ermöglichte es, flexibel auf sich ändernde Anforderungen und Herausforderungen während der Entwickl-
ung zu reagieren. Der Arbeitsprozess war stark \hyperlink{iterativ}{\textcolor{blue}{iterativ}} geprägt, mit regelmäßigen Rücksprachen und enger Zusammenarbeit mit den Kollegen.

Der \hyperlink{iterativ}{\textcolor{blue}{iterative}} Zyklus bestand darin, Arbeitsschritte zu planen, umzusetzen, zu evaluieren und anhand des erhaltenen Feedbacks kontinuierlich zu verbessern. Diese kurzen Iterationszyklen erlaubten es, Ergebnisse frühzeitig zu präsentieren und Feedback direkt in die nächste Phase einfließen zu lassen. Dies förderte nicht nur eine hohe Anpassungsfähigkeit, sondern auch eine kontinuierliche Verbesserung der Qualität des Projekts.

Die \hyperlink{agil}{\textcolor{blue}{agile}} Vorgehensweise half dabei, schnelle Anpassungen an neuen Anforderungen vorzunehmen, die in den regelmäßigen Besprechungen und durch Feedback der Kollegen eingebracht wurden. Durch den Einsatz von Tools wie \hyperlink{MicrosoftTeams}{\textcolor{blue}{Microsoft Teams}} und \hyperlink{Jira}{\textcolor{blue}{Jira}} wurde die Zusammenarbeit und das Projektmanageme-
nt unterstützt, sodass der Fortschritt transparent blieb und die Arbeitsschritte effektiv geplant und dokumentiert werden konnten.
