% !TEX root = ../Projektdokumentation.tex
\section{Einleitung}
\label{sec:Einleitung}

Diese Projektdokumentation beschreibt den Ablauf des Projektes „Implementierung automatisierter \hyperlink{Onboarding}{\textcolor{AOBlau}{Onboarding}}- und Ressourcenmanagement-Prozesse zur Optimierung der \hyperlink{UserExperience}{\textcolor{AOBlau}{User Experience}} für neue Entwickler bei Touristik Union International (TUI) Group“. 

Die vorliegende Dokumentation wurde projektbegleitend erstellt und dient der Abschlussprüfung im Ausbildungsberuf Fachinformatiker Anwendungsentwicklung. Sie erläutert die Ziele, den Ablauf und die Ergebnisse des Projektes zur Automatisierung des \hyperlink{Onboarding}{\textcolor{AOBlau}{Onboarding}}-Prozesses für neue Entwickler. Des Weiteren werden die eingesetzten Technologien und die Interaktionen mit verschiedenen Systemen beschrieben. 

Die \textcolor{AOBlau}{blau} markierten Begriffe werden nicht direkt im Text erklärt, sondern im angehängten Glossar.

\subsection{Projektumfeld} 
\label{sec:Projektumfeld}

Die \hyperlink{TUI}{\textcolor{AOBlau}{TUI} InfoTec GmbH} ist eine Tochtergesellschaft der \hyperlink{TUI}{\textcolor{AOBlau}{TUI}} AG, einem weltweit führenden Unternehm-
en im Bereich Tourismus und Reisen. Als interner IT-Dienstleister übernimmt die \hyperlink{TUI}{\textcolor{AOBlau}{TUI}} InfoTec GmbH die IT-Betreuung und -Optimierung für die gesamte \hyperlink{TUI}{\textcolor{AOBlau}{TUI}} AG. Sie ist verantwortlich für die Bereitstellu-
ng und Wartung der IT-Infrastruktur sowie für die Entwicklung und Implementierung von Softwarelös-
ungen, die die Geschäftsprozesse innerhalb des Konzerns unterstützen.

Die \hyperlink{TUI}{\textcolor{AOBlau}{TUI}} InfoTec GmbH beschäftigt zurzeit etwas mehr als 600 Mitarbeiter\footnote{Stand 2021.}, die in unterschiedlichen Bereichen der IT tätig sind. Die \textbf{Shared Services} Abteilung, die als Auftraggeber für dieses Projekt fungiert, bietet zentrale IT-Services für verschiedene Abteilungen des Unternehmens an. Dabei konzent-
riert sich diese Abteilung besonders darauf, den Entwicklern eine optimale Arbeitsumgebung bereitzus-
tellen, um sie von zeitaufwendigen, wiederkehrenden Aufgaben zu entlasten und ihre Produktivität zu steigern.

Die gesamte \hyperlink{TUI}{\textcolor{AOBlau}{TUI}} Group beschäftigt hingegen über 65.413 Mitarbeiter\footnote{Stand 2023.}, was die Größe und die Bedeut-
ung des Unternehmens im globalen Tourismussektor unterstreicht.


\subsection{Projektziel} 
\label{sec:Projektziel}

Ziel dieses Projekts ist es, die Effizienz und \hyperlink{UserExperience}{\textcolor{AOBlau}{User Experience}} beim \hyperlink{Onboarding}{\textcolor{AOBlau}{Onboarding}} neuer Entwickler durch automatisierte Prozesse zu verbessern. Der bisherige Onboarding-Prozess erfolgt größtenteils manuell und umfasst häufig den zeitaufwändigen Schritt, neuen Entwicklern die Informationen zu internen Abläufen, Dokumentationsstandorten und weiteren wichtigen Ressourcen manuell im Laufe des Einstiegs zu vermitteln. Zusätzlich müssen neue Entwickler manuell in die relevanten \hyperlink{MicrosoftTeams}{\textcolor{AOBlau}{Microsoft Teams}} Gruppen aufgenommen werden. 

Im Rahmen des Projekts wird eine Anwendung entwickelt, die auf \hyperlink{GitLab}{\textcolor{AOBlau}{GitLab}} Events reagiert und eine automatisierte Willkommens-E-Mail an neue Entwickler sendet. Diese E-Mail enthält wichtige Informa-
tionen zu internen Prozessen, Dokumentationsstandorten und weiteren zentralen Ressourcen, sodass neue Entwickler direkt beim Einstieg eine strukturierte Übersicht erhalten und effizient auf die notwend-
igen Informationen zugreifen können. Zusätzlich fügt die Anwendung die neuen Entwickler automatisiert den passenden \hyperlink{MicrosoftTeams}{\textcolor{AOBlau}{Microsoft Teams}}-Gruppen hinzu. Diese Automatisierung steigert die Produktivität neuer Mitarbeiter und reduziert zugleich den manuellen Aufwand für das IT-Team erheblich.

Das Projektergebnis soll eine Lösung sein, die sowohl den aktuellen Anforderungen gerecht wird als auch einfach zu verwalten und skalierbar ist, um flexibel auf zukünftige Anforderungen reagieren zu können.

\subsection{Projektbegründung} 
\label{sec:Projektbegruendung}

Der aktuelle \hyperlink{Onboarding}{\textcolor{AOBlau}{Onboarding}}-Prozess für neue Entwickler ist größtenteils manuell und erfordert erheblichen Zeitaufwand, was das IT-Team zusätzlich belastet und den Einstieg der neuen Mitarbeiter erschwert. Neue Entwickler werden oft nicht sofort in die richtigen Teams und Kommunikationskanäle integriert und erhalten möglicherweise nicht alle relevanten Informationen zum Einstieg in die Entwicklungsprozes-
se der \hyperlink{TUI}{\textcolor{AOBlau}{TUI}}. Dies kann zu Verzögerungen führen, die den Produktivitätseintritt der Entwickler behindern. Ein automatisierter Prozess würde sicherstellen, dass jeder neue Entwickler sofort alle nötigen Ressourcen und Teammitgliedschaften erhält und gleichzeitig unnötige manuelle Arbeit für das IT-Team entfällt.

Die Automatisierung des \hyperlink{Onboarding}{\textcolor{AOBlau}{Onboarding}}-Prozesses bietet somit klare Vorteile:
- \textbf{Zeitersparnis}: Die Automatisierung reduziert die Zeit, die das IT-Team für manuelle Aufgaben aufwenden muss. Neue Entwickler können ohne Verzögerung in die relevanten Gruppen aufgenommen und erhalten automatisch alle notwendigen Informationen.
- \textbf{Kosteneffizienz}: Durch die Reduktion des manuellen Aufwands werden nicht nur Fehler vermieden, sondern auch Ressourcen effizienter eingesetzt. Das IT-Team kann seine Kapazitäten für wichtigere Aufgaben nutzen.


\subsection{Projektschnittstellen} 
\label{sec:Projektschnittstellen}

Die entwickelte Anwendung interagiert mit verschiedenen Systemen, um den \hyperlink{Onboarding}{\textcolor{AOBlau}{Onboarding}}-Prozess für neue Entwickler zu automatisieren. Ein zentraler Bestandteil ist die Integration mit \hyperlink{GitLab}{\textcolor{AOBlau}{GitLab}}, das bei \hyperlink{TUI}{\textcolor{AOBlau}{TUI}} sowohl für die Versionskontrolle als auch für \hyperlink{CI}{\textcolor{AOBlau}{Continuous Integration}} und \hyperlink{CD}{\textcolor{AOBlau}{Continuous Deployment}} (CI/CD) genutzt wird. Um die Automatisierung zu ermöglichen, werden \hyperlink{Schnittstelle}{\textcolor{AOBlau}{Schnittstellen}} in der Programmiersprache \hyperlink{Go}{\textcolor{AOBlau}{Go}} entwickelt, die auf \hyperlink{GitLab}{\textcolor{AOBlau}{GitLab Events}} reagieren und die entsprechenden Prozesse auslösen.

Das Projekt wurde durch den Head of Technology der \textbf{Shared Services} Abteilung genehmigt. Innerhalb der Abteilung standen zudem Mitarbeiter zur Verfügung, um bei Rückfragen zu unterstützen und regelmäßiges Feedback während der Entwicklung zu geben.

Die \textbf{Benutzer} der Anwendung sind neue Entwickler, die automatisch in den \hyperlink{Onboarding}{\textcolor{AOBlau}{Onboarding}}-Prozess integriert werden, um eine reibungslose Einarbeitung und den Zugang zu den erforderlichen Ressourcen zu gewährleisten.

\subsection{Projektabgrenzung} 
\label{sec:Projektabgrenzung}

Die aktuelle Refaktorisierung des bestehenden Systems ist nicht Teil dieses Projekts. Zukünftige Anpassungen des aktuellen Systems sind erforderlich, um die \hyperlink{GitLabSystemhooks}{\textcolor{AOBlau}{GitLab-Systemhooks}}, die wir derzeit verwenden, entsprechend zu überarbeiten und anzupassen. Diese Änderungen werden separat behandelt und sind nicht im Rahmen der gegenwärtigen Automatisierungsprojekte enthalten.
