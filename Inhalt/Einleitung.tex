% !TEX root = ../Projektdokumentation.tex
\section{Einleitung}
\label{sec:Einleitung}

\subsection{Projektumfeld} 
\label{sec:Projektumfeld}

Die \textbf{TUI InfoTec GmbH} ist eine Tochtergesellschaft der \textbf{TUI AG}, einem weltweit führenden Anbieter im Bereich Tourismus und Reisen. Die TUI InfoTec GmbH ist für die gesamte interne IT-Betreuung der TUI AG zuständig und spielt eine Schlüsselrolle bei der Gestaltung und Optimierung der IT-Infrastruktur sowie der Bereitstellung von Softwarelösungen, die die Geschäftsprozesse innerhalb des TUI-Konzerns unterstützen und effizient gestalten.

Die TUI InfoTec GmbH beschäftigt mehrere hundert Mitarbeiter, die in unterschiedlichen Bereichen der IT tätig sind. Insbesondere konzentriert sich die \textbf{Shared Services Abteilung}, die den Onboarding-Prozess für Entwickler optimieren möchte, auf die Bereitstellung von zentralen IT-Services für alle anderen Abteilungen. Ein bedeutender Teil dieser Aufgabe ist es, den Entwicklern eine optimale Arbeitsumgebung zu schaffen, sodass sie sich auf ihre eigentlichen Aufgaben konzentrieren können, ohne durch wiederkehrende und zeitraubende Aufgaben abgelenkt zu werden. Die Shared Services Abteilung fungiert daher als Projektauftraggeber und hat mir als Projektverantwortlichem den Auftrag erteilt, die Onboarding-Prozesse für Entwickler zu automatisieren.

\subsection{Projektziel} 
\label{sec:Projektziel}

Das Ziel dieses Projekts ist die Automatisierung des Onboarding-Prozesses für neue Entwickler innerhalb des Unternehmens. Der derzeitige Onboarding-Prozess ist weitgehend manuell und beinhaltet zeitaufwändige Schritte wie das Hinzufügen neuer Entwickler zu Microsoft Teams Gruppen, das Versenden von E-Mails mit wichtigen Informationen zu internen Prozessen und Dokumentationen sowie das Bereitstellen von Zugang zu GitLab-Repositories und weiteren Entwicklungsressourcen. Diese Aufgaben erfolgen derzeit durch manuelle Eingriffe, die nicht nur Zeit kosten, sondern auch potenziell zu Fehlern führen können, wenn sie nicht korrekt durchgeführt werden.

Durch das Projekt soll eine Anwendung entwickelt werden, die automatisch auf Ereignisse innerhalb von \textbf{GitLab} reagiert, um neue Entwickler in die relevanten Microsoft Teams Gruppen zu integrieren und ihnen automatisch eine E-Mail mit wichtigen Informationen zu internen Prozessen und Dokumentationen zu senden. Diese Automatisierung soll den Einstieg für neue Entwickler erheblich vereinfachen, ihre Produktivität steigern und gleichzeitig den manuellen Aufwand für das IT-Team minimieren. Das Endziel ist es, eine Lösung zu schaffen, die sowohl den aktuellen Anforderungen gerecht wird als auch skalierbar und erweiterbar ist, um auf zukünftige Anforderungen flexibel reagieren zu können.

\subsection{Projektbegründung} 
\label{sec:Projektbegruendung}

Der derzeit manuelle Onboarding-Prozess für neue Entwickler ist sowohl zeitaufwendig als auch fehleranfällig. Neue Entwickler werden oft nicht sofort in die richtigen Teams und Kommunikationskanäle integriert und erhalten möglicherweise nicht alle relevanten Informationen zum Einstieg in die Entwicklungsprozesse bei TUI. Dies kann zu Verzögerungen führen, die den Produktivitätseintritt der Entwickler behindern. Ein automatisierter Prozess würde sicherstellen, dass jeder neue Entwickler sofort alle nötigen Ressourcen und Teammitgliedschaften erhält und gleichzeitig unnötige manuelle Arbeit für das IT-Team entfällt.

Die Automatisierung des Onboarding-Prozesses bietet somit klare Vorteile:
- \textbf{Zeitersparnis}: Die Automatisierung reduziert die Zeit, die das IT-Team für manuelle Aufgaben aufwenden muss. Neue Entwickler können ohne Verzögerung in die relevanten Gruppen aufgenommen und erhalten automatisch alle notwendigen Informationen.
- \textbf{Fehlerreduktion}: Durch die Automatisierung wird das Risiko von menschlichen Fehlern verringert, die bei manuellen Prozessen auftreten können, wie etwa das Vergessen, einen Entwickler in eine wichtige Gruppe zu integrieren.
- \textbf{Kosteneffizienz}: Durch die Reduktion des manuellen Aufwands werden nicht nur Fehler vermieden, sondern auch Ressourcen effizienter eingesetzt. Das IT-Team kann seine Kapazitäten für wichtigere Aufgaben nutzen.

Die Motivation hinter dem Projekt ist, die Einarbeitungszeit neuer Entwickler zu verkürzen und gleichzeitig eine höhere Konsistenz und Qualität im Onboarding-Prozess zu gewährleisten. So wird der Einstieg für neue Entwickler erleichtert, und sie können schneller produktiv arbeiten. Gleichzeitig wird die Effizienz des IT-Teams gesteigert, da der manuelle Aufwand durch Automatisierung entfällt.

\subsection{Projektschnittstellen} 
\label{sec:Projektschnittstellen}

Die entwickelte Anwendung wird mit mehreren Systemen und Plattformen interagieren, um den Onboarding-Prozess zu automatisieren. Ein zentraler Bestandteil dieser Lösung ist die Integration mit \textbf{GitLab}, da GitLab bei TUI als Hauptplattform für die Versionskontrolle und die kontinuierliche Integration genutzt wird. Die Anwendung wird auf GitLab-Events reagieren, wie zum Beispiel das Hinzufügen eines neuen Entwicklers zu einem Projekt oder das Erstellen eines neuen Repositories, um entsprechende Automatisierungen auszulösen.

Zusätzlich wird die Anwendung in \textbf{Microsoft Teams} integriert, da Teams die zentrale Kommunikationsplattform für die Entwickler bei TUI ist. Über Microsoft Teams werden neue Entwickler automatisch in die relevanten Teams-Gruppen aufgenommen, die für die Zusammenarbeit und Kommunikation innerhalb des Projekts erforderlich sind.

Neben GitLab und Microsoft Teams wird auch \textbf{Jira}, das Projektmanagement-Tool bei TUI, eine Rolle spielen, um den Fortschritt des Onboardings zu verfolgen und Aufgaben für neue Entwickler zu erstellen. Jira wird als Schnittstelle verwendet, um sicherzustellen, dass alle Schritte des Onboarding-Prozesses dokumentiert und nachvollziehbar sind.

Der Auftraggeber dieses Projekts ist die \textbf{Shared Services Abteilung} der TUI InfoTec GmbH. Diese Abteilung stellt die finanziellen Mittel sowie die Ressourcen zur Verfügung und ist für die Genehmigung des Projekts zuständig. Das Ergebnis des Projekts wird sowohl dem Auftraggeber als auch dem Entwicklungsteam präsentiert, um sicherzustellen, dass die Lösung den Anforderungen und Erwartungen entspricht.

Die \textbf{Benutzer} der Anwendung sind hauptsächlich die neuen Entwickler, die von der automatisierten Lösung profitieren werden. Die IT-Administratoren und das Entwicklungsteam werden die Lösung zur Verwaltung und Überwachung des Onboarding-Prozesses nutzen.

\subsection{Projektabgrenzung} 
\label{sec:Projektabgrenzung}

Dieses Projekt fokussiert sich ausschließlich auf die Automatisierung des Onboarding-Prozesses für neue Entwickler. Es umfasst nicht die Automatisierung des gesamten IT-Onboarding-Prozesses bei TUI, also beispielsweise nicht die Bereitstellung von Hardware, die Integration in andere nicht entwicklungsrelevante IT-Systeme oder die Verwaltung von Zugriffsrechten für nicht-technische Teams. Diese Aufgaben sind nicht Teil dieses Projekts und würden in separaten Projekten behandelt werden.

Zudem werden in diesem Projekt keine erweiterten Sicherheitsfunktionen wie beispielsweise ein Host-Checker implementiert, um Clients auf Sicherheitslücken zu überprüfen. Solche Funktionen könnten in zukünftigen Projekten als Erweiterungen hinzugefügt werden, sind aber nicht Bestandteil dieses aktuellen Projekts.
