% !TEX root = ../Projektdokumentation.tex
\section{Einleitung}
\label{sec:Einleitung}

Diese Projektdokumentation beschreibt den Ablauf des Projektes „Implementierung automatisierter \hyperlink{Onboarding}{\textcolor{blue}{Onboarding}}- und Ressourcenmanagement-Prozesse zur Optimierung der \hyperlink{UserExperience}{\textcolor{blue}{User Experience}} für neue Entwickler bei \ac{TUI} Group“. 

Die vorliegende Dokumentation wurde projektbegleitend erstellt und dient der Abschlussprüfung im Ausbildungsberuf Fachinformatiker Anwendungsentwicklung. Sie erläutert die Ziele, den Ablauf und die Ergebnisse des Projektes zur Automatisierung des \hyperlink{Onboarding}{\textcolor{blue}{Onboarding}}-Prozesses für neue Entwickler. Des Weiteren werden die eingesetzten Technologien und die Interaktionen mit verschiedenen Systemen beschrieben. 

Die \textcolor{blue}{blau} markierten Begriffe werden nicht direkt im Text erklärt, sondern im angehängten Glossar.

\subsection{Projektumfeld} 
\label{sec:Projektumfeld}

Die \textbf{\hyperlink{TUI}{\textcolor{blue}{TUI}} InfoTec GmbH} ist eine Tochtergesellschaft der \textbf{\hyperlink{TUI}{\textcolor{blue}{TUI}} AG}, einem weltweit führenden Unternehmen im Bereich Tourismus und Reisen. Als interner IT-Dienstleister übernimmt die \hyperlink{TUI}{\textcolor{blue}{TUI}} InfoTec GmbH die IT-Betreuung und -Optimierung für die gesamte \hyperlink{TUI}{\textcolor{blue}{TUI}} AG. Sie ist verantwortlich für die Bereitstellung und Wartung der IT-Infrastruktur sowie für die Entwicklung und Implementierung von Softwarelösungen, die die Geschäftsprozesse innerhalb des Konzerns unterstützen.

Die \hyperlink{TUI}{\textcolor{blue}{TUI}} InfoTec GmbH beschäftigt zurzeit etwas mehr
als 600 Mitarbeiter, die in unterschiedlichen Bereichen der IT tätig sind.
Die \textbf{Shared Services Abteilung}, die als Auftraggeber für dieses Projekt fungiert, 
bietet zentrale IT-Services für verschiedene Abteilungen des Unternehmens an.
Dabei konzentriert sich diese Abteilung besonders darauf, den Entwicklern eine
optimale Arbeitsumgebung bereitzustellen, um sie von zeitaufwendigen, wiederkehrenden
Aufgaben zu entlasten und ihre Produkti-
vität zu steigern.

\subsection{Projektziel} 
\label{sec:Projektziel}

Das Ziel dieses Projekts ist es, die Effizienz und \hyperlink{UserExperience}{\textcolor{blue}{User Experience}} beim \hyperlink{Onboarding}{\textcolor{blue}{Onboarding}} neuer Entwickler durch automatisierte Prozesse zu verbessern. Der aktuelle Onboarding-Prozess ist überwiegend manuell und erfordert zeitaufwändige Schritte wie das Hinzufügen neuer Entwickler zu \textbf{\hyperlink{MicrosoftTeams}{\textcolor{blue}{Microsoft Teams}}} Gruppen, das Versenden von E-Mails mit wichtigen Informationen zu internen Prozessen und das Bereitstellen von Zugriffen auf \hyperlink{GitLab}{\textcolor{blue}{GitLab}}-Repositories sowie weitere Entwicklungsressourcen. Diese manu-
ellen Tätigkeiten sind nicht nur zeitintensiv, sondern bergen auch das Risiko von Fehlern.

Im Rahmen dieses Projekts wird eine Anwendung entwickelt, die auf \hyperlink{GitLab}{\textcolor{blue}{GitLab}}-Events reagiert und automatisch neue Entwickler in die relevanten \textbf{\hyperlink{MicrosoftTeams}{\textcolor{blue}{Microsoft Teams}}} Gruppen integriert. Zusätzlich werden E-Mails mit zentralen Informationen zu internen Abläufen und Dokumentationsstandorten verschickt, um den Einstieg für neue Entwickler zu erleichtern. Dadurch wird die Produktivität neuer Mitarbeiter gesteigert und gleichzeitig der manuelle Aufwand für das IT-Team deutlich reduziert.

Am Ende des Projekts soll eine Lösung stehen, die sowohl den aktuellen Anforderungen gerecht wird als auch einfach verwaltbar und skalierbar ist, um auf zukünftige Bedürfnisse flexibel reagieren zu können.

\subsection{Projektbegründung} 
\label{sec:Projektbegruendung}

Der derzeit manuelle \hyperlink{Onboarding}{\textcolor{blue}{Onboarding}}-Prozess für neue Entwickler ist sowohl zeitaufwendig als auch fehlera-
nfällig. Neue Entwickler werden oft nicht sofort in die richtigen Teams und Kommunikationskanäle integriert und erhalten möglicherweise nicht alle relevanten Informationen zum Einstieg in die Entwic-
klungsprozesse bei \hyperlink{TUI}{\textcolor{blue}{TUI}}. Dies kann zu Verzögerungen führen, die den Produktivitätseintritt der Entwic-
kler behindern. Ein automatisierter Prozess würde sicherstellen, dass jeder neue Entwickler sofort alle nötigen Ressourcen und Teammitgliedschaften erhält und gleichzeitig unnötige manuelle Arbeit für das IT-Team entfällt.

Die Automatisierung des \hyperlink{Onboarding}{\textcolor{blue}{Onboarding}}-Prozesses bietet somit klare Vorteile:
- \textbf{Zeitersparnis}: Die Automatisierung reduziert die Zeit, die das IT-Team für manuelle Aufgaben aufwenden muss. Neue Entwickler können ohne Verzögerung in die relevanten Gruppen aufgenommen und erhalten automatisch alle notwendigen Informationen.
- \textbf{Kosteneffizienz}: Durch die Reduktion des manuellen Aufwands werden nicht nur Fehler vermieden, sondern auch Ressourcen effizienter eingesetzt. Das IT-Team kann seine Kapazitäten für wichtigere Aufgaben nutzen.

Die Motivation hinter dem Projekt ist, die Einarbeitungszeit neuer Entwickler zu verkürzen und gleichzeitig eine höhere Konsistenz und Qualität im \hyperlink{Onboarding}{\textcolor{blue}{Onboarding}}-Prozess zu gewährleisten. So wird der Einstieg für neue Entwickler erleichtert, und sie können schneller produktiv arbeiten. 

\subsection{Projektschnittstellen} 
\label{sec:Projektschnittstellen}

Die entwickelte Anwendung interagiert mit verschiedenen Systemen, um den \hyperlink{Onboarding}{\textcolor{blue}{Onboarding}}-Prozess für neue Entwickler zu automatisieren. Ein zentraler Bestandteil ist die Integration mit \hyperlink{GitLab}{\textcolor{blue}{GitLab}}, das bei \hyperlink{TUI}{\textcolor{blue}{TUI}} sowohl für die Versionskontrolle als auch für \hyperlink{CI}{\textcolor{blue}{Continuous Integration}} und \hyperlink{CD}{\textcolor{blue}{Continuous Deployment}} (CI/CD) genutzt wird. Um die Automatisierung zu ermöglichen, werden \hyperlink{Schnittstelle}{\textcolor{blue}{Schnittstellen}} in der Programmiersprache \hyperlink{Go}{\textcolor{blue}{Go}} entwickelt, die auf \hyperlink{GitLab}{\textcolor{blue}{GitLab}}-Ereignisse reagieren und die entsprechenden Prozesse auslösen.

Das Projekt wurde durch den \textbf{Head of Technology} der \textbf{Shared Services Abteilung} genehmigt. Innerhalb der Abteilung standen zudem Mitarbeiter zur Verfügung, um bei Rückfragen zu unterstützen und regelmäßiges Feedback während der Entwicklung zu geben.

Die \textbf{Benutzer} der Anwendung sind neue Entwickler, die automatisch in den \hyperlink{Onboarding}{\textcolor{blue}{Onboarding}}-Prozess integriert werden, um eine reibungslose Einarbeitung und den Zugang zu den erforderlichen Ressourcen zu gewährleisten.

\subsection{Projektabgrenzung} 
\label{sec:Projektabgrenzung}

Die aktuelle Refaktorisierung des bestehenden Systems ist nicht Teil dieses Projekts. Zukünftige Anpassungen des aktuellen Systems sind erforderlich, um die \hyperlink{GitLabSystemhooks}{\textcolor{blue}{GitLab-Systemhooks}}, die wir derzeit verwenden, entsprechend zu überarbeiten und anzupassen. Diese Änderungen werden separat behandelt und sind nicht im Rahmen der gegenwärtigen Automatisierungsprojekte enthalten.
