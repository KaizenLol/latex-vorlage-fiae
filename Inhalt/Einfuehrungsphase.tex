% !TEX root = ../Projektdokumentation.tex
\section{Einführungsphase}
\label{sec:Einfuehrungsphase}

\subsection{EKS-Deployment}
\label{sec:EKSDeployment}

Die GitLab-Systemhook und der Dev-Kickstarter wurden im Rahmen einer automatisierten CI/CD-Pipeline containerisiert und anschließend in die GitLab-Registry hochgeladen. Um die Anwendungen im Kubernetes-Cluster (EKS) bereitzustellen, wurde eine \texttt{yaml}-Konfigurationsdatei erstellt. Diese Datei definiert alle notwendigen Ressourcen und Parameter, wie z.B. die Container-Images, Ports, Umgebungsvariablen und Ingress-Konfigurationen.

Die fertige \texttt{yaml}-Datei wurde dann in ein internes Repository commitet, welches für alle unsere EKS-Deployments verwendet wird. Sobald der Commit erfolgt ist, wird das Deployment im Cluster automatisch ausgelöst und die Anwendungen werden dort bereitgestellt. 

Ein Beispiel für die verwendete \texttt{yaml}-Konfiguration kann im Anhang \ref{app:yamlFile} eingesehen werden.

\subsection{AWS-Deployment}
\label{sec:AWSDeployment}

Für das AWS-Deployment wurden die erforderlichen Infrastrukturkomponenten in der AWS-Umgebung konfiguriert. Dazu gehören Amazon SNS (Simple Notification Service) zur Kommunikation und Amazon SQS (Simple Queue Service) zur Nachrichtenverarbeitung. 

Die Bereitstellung dieser Ressourcen erfolgte mithilfe von Infrastructure as Code (IaC) mit Terraform. Dabei wurden SNS-Topics und SQS-Queues eingerichtet, um die Kommunikation zwischen den Diensten zu ermöglichen. Eine Filter-Policy wurde verwendet, um sicherzustellen, dass nur spezifische Ereignisse, wie das Erstellen eines Benutzers (eventName: \texttt{user\_create}), an die SQS-Queue weitergeleitet werden. 

Zusätzlich wurde eine IAM-Policy erstellt, die dem Dev-Kickstarter die erforderlichen Berechtigungen gewährt, um E-Mails über Amazon SES (Simple Email Service) zu senden und Nachrichten aus der SQS-Queue zu empfangen. 

Die Terraform-Ressourcen sind in einem internen Repository gespeichert. Ein Beispiel für die verwendeten Terraform-Ressourcen kann im Anhang \ref{app:terraform} eingesehen werden.
