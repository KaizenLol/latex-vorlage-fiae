% !TEX root = ../Projektdokumentation.tex
\section{Einführungsphase}
\label{sec:Einfuehrungsphase}

\subsection{EKS-Deployment}
\label{sec:EKSDeployment}

Die \hyperlink{GitLabSystemhooks}{\textcolor{AOBlau}{GitLab-Systemhook}} und der Dev-Kickstarter wurden im Rahmen einer automatisierten \hyperlink{CI}{\textcolor{AOBlau}{CI/CD}}-Pipeline containerisiert und anschließend in die \hyperlink{GitLab}{\textcolor{AOBlau}{GitLab-Registry}} hochgeladen. Um die Anwendungen im \hyperlink{EKS}{\textcolor{AOBlau}{Kubernetes-Cluster (EKS)}} bereitzustellen, wurde eine \hyperlink{YAML}{\textcolor{AOBlau}{yaml}}-Konfigurationsdatei erstellt. Diese Datei definiert alle notwendigen Ressourcen und Parameter, wie z.B. die \hyperlink{ContainerImage}{\textcolor{AOBlau}{Container-Images}}, \hyperlink{Ports}{\textcolor{AOBlau}{Ports}}, \hyperlink{Umgebungsvariablen}{\textcolor{AOBlau}{Umgebungsvariablen}} und \hyperlink{KubernetesIngress}{\textcolor{AOBlau}{Ingress-Konfigurationen}}.

Die fertige \hyperlink{GitCommit}{\textcolor{AOBlau}{yaml}}-Datei wurde dann in ein internes Repository commitet, welches für alle unsere EKS-Deployments verwendet wird. Sobald der \hyperlink{GitCommit}{\textcolor{AOBlau}{Commit}} erfolgt ist, wird das Deployment im Cluster automatisch ausgelöst und die Anwendungen werden dort bereitgestellt. 

Ein Beispiel für die verwendete \hyperlink{YAML}{\textcolor{AOBlau}{yaml}}-Konfiguration kann im \Anhang{app:yamlFile} eingesehen werden.

\subsection{AWS-Deployment}
\label{sec:AWSDeployment}

Für das AWS-Deployment wurden die erforderlichen Infrastrukturkomponenten in der AWS-Umgebung konfiguriert. Dazu gehören \hyperlink{SNS}{\textcolor{AOBlau}{Amazon SNS}} zur Kommunikation und \hyperlink{SQS}{\textcolor{AOBlau}{Amazon SQS}} zur Nachrichtenverarb-
eitung. 

Die Bereitstellung dieser Ressourcen erfolgte mithilfe von \hyperlink{InfrastructureAsCode}{\textcolor{AOBlau}{Infrastructure as Code}} mit \hyperlink{Terraform}{\textcolor{AOBlau}{Terraform}}.\footnote{Spacelift - Infrastructure as Code, \cite{Iac}.} Dabei wurden \hyperlink{SNS}{\textcolor{AOBlau}{SNS-Topics}} und \hyperlink{SQS}{\textcolor{AOBlau}{SQS-Queues}} eingerichtet, um die Kommunikation zwischen den Diensten zu ermöglichen. Eine \hyperlink{FilterPolicies}{\textcolor{AOBlau}{Filter-Policy}} wurde verwendet, um sicherzustellen, dass nur spezifische Ereignisse, wie das Erstellen eines Benutzers (eventName: \texttt{user\_create}), an die \hyperlink{SQS}{\textcolor{AOBlau}{SQS-Queue}} weitergeleitet werden. 

Zusätzlich wurde eine \hyperlink{SES}{\textcolor{AOBlau}{IAM-Policy}} erstellt, die dem Dev-Kickstarter die erforderlichen Berechtigungen gewährt, um E-Mails über \hyperlink{SES}{\textcolor{AOBlau}{Amazon SES}} zu senden und Nachrichten aus der \hyperlink{SQS}{\textcolor{AOBlau}{SQS-Queue}} zu empfangen. 

Die \hyperlink{Terraform}{\textcolor{AOBlau}{Terraform}}-Ressourcen sind in einem internen Repository gespeichert. Ein Beispiel für die verwendet-
en \hyperlink{Terraform}{\textcolor{AOBlau}{Terraform}}-Ressourcen kann im \Anhang{app:terraform} eingesehen werden.
