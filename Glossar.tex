% Glossar
\begin{description}
    \item[\hypertarget{MicrosoftTeams}{Microsoft Teams}] Tool für Teamkommunikation.
    \item[\hypertarget{CI}{CI}] Continuous Integration – regelmäßige Codeintegration.
    \item[\hypertarget{CD}{CD}] Continuous Deployment – automatisierte Bereitstellung.
    \item[\hypertarget{CSS}{CSS}] Sprache zur Gestaltung von Weblayouts.
    \item[\hypertarget{InlineCSS}{Inline CSS}] CSS-Stile direkt im HTML-Element.
    \item[\hypertarget{Stylesheets}{Stylesheets}] Dokumente mit CSS-Regeln.
    \item[\hypertarget{Kubernetes}{Kubernetes}] System zur Container-Orchestrierung.
    \item[\hypertarget{YAML}{YAML}] Lesbares Datenformat für Konfigurationen.
    \item[\hypertarget{ContainerImage}{Container Image}] Paket für ausführbare Anwendungen.
    \item[\hypertarget{Ports}{Ports}] Virtuelle Kommunikationspunkte in Netzwerken.
    \item[\hypertarget{Umgebungsvariablen}{Umgebungsvariablen}] Werte für Prozessumgebungen.
    \item[\hypertarget{KubernetesIngress}{Kubernetes Ingress}] Regelt HTTP/HTTPS-Zugriff im Cluster.
    \item[\hypertarget{Deployment}{Deployment}] Sicherstellung der Pod-Ausführung in Kubernetes.
    \item[\hypertarget{InfrastructureAsCode}{Infrastructure as Code}] Infrastrukturverwaltung per Code.
    \item[\hypertarget{Terraform}{Terraform}] Tool für Infrastrukturverwaltung als Code.
    \item[\hypertarget{Go}{Go}] Programmiersprache von Google.
    \item[\hypertarget{HTML}{HTML}] Standardsprache für Webdokumente.
    \item[\hypertarget{GitLab}{GitLab}] Plattform für Versionskontrolle und DevOps.
    \item[\hypertarget{OneSource}{OneSource}] Unternehmens-GitLab-Instanz.
    \item[\hypertarget{Runway}{Runway}] Interne Entwicklerplattform und Dokumentationsquelle.
    \item[\hypertarget{GitLabSystemhooks}{GitLab-Systemhooks}] Reaktionen auf GitLab-Events.
    \item[\hypertarget{GitLabEvent}{GitLab Event}] Events innerhalb von GitLab.
    \item[\hypertarget{GitCommit}{Git Commit}] Speicherung von Codeänderungen.
    \item[\hypertarget{Onboarding}{Onboarding}] Integration neuer Mitarbeiter.
    \item[\hypertarget{UserExperience}{User Experience (UX)}] Benutzererfahrung bei der Produktnutzung.
    \item[\hypertarget{Schnittstelle}{Schnittstelle}] Kommunikationspunkt zwischen Systemen.
    \item[\hypertarget{Automatisierung}{Automatisierung}] Automatisierte Prozessausführung.
    \item[\hypertarget{Jira}{Jira}] Projektmanagement-Software.
    \item[\hypertarget{agil}{agil}] Flexibler Entwicklungsansatz.
    \item[\hypertarget{iterativ}{iterativ}] Prozess mit wiederholten Zyklen.
    \item[\hypertarget{Microservices}{Microservices}] Architektur aus unabhängigen Diensten.
    \item[\hypertarget{CodeReview}{Code Review}] Überprüfung von Quellcode.
    \item[\hypertarget{UnitTests}{Unit Tests}] Tests einzelner Softwarekomponenten.
    \item[\hypertarget{IntegrationTests}{Integration Tests}] Tests des Zusammenspiels von Komponenten.
    \item[\hypertarget{BestPractices}{Best Practices}] Erprobte Verfahren in der Softwareentwicklung.
    \item[\hypertarget{SQS}{SQS}] AWS-Dienst für Nachrichtenwarteschlangen.
    \item[\hypertarget{SNS}{SNS}] AWS-Dienst für Benachrichtigungen.
    \item[\hypertarget{AWS}{AWS}] Cloud-Plattform von Amazon.
    \item[\hypertarget{SES}{SES}] AWS-E-Mail-Dienst.
    \item[\hypertarget{EKS}{EKS}] Verwalteter Kubernetes-Dienst von AWS.
    \item[\hypertarget{HTTP}{HTTP}] Protokoll für Datenübertragung im Web.
    \item[\hypertarget{HTTPPOST}{HTTP POST}] Methode zum Senden von Daten an Server.
    \item[\hypertarget{FilterPolicies}{Filter Policies}] Filter für Nachrichtenattribute.
    \item[\hypertarget{GraphClient}{Graph Client}] Client für Microsoft Graph API.
    \item[\hypertarget{Payload}{Payload}] Daten in einer Nachricht.
    \item[\hypertarget{MessageAttributes}{Message Attributes}] Zusätzliche Nachrichteninformationen.
    \item[\hypertarget{SQSConsumer}{SQS Consumer}] Komponente zur Verarbeitung von SQS-Nachrichten.
    \item[\hypertarget{Flag}{Flag}] Kennzeichnung für Zustände oder Optionen.
    \item[\hypertarget{Endpoints}{Endpunkte}] URLs für API-Kommunikation.
\end{description}
