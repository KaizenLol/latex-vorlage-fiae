% Glossar
\newpage
\section*{Glossar}
\begin{description}
    \item[\hypertarget{MicrosoftTeams}{Microsoft Teams}] Kommunikations-Tool, das für Teamarbeit und Kommunikation verwendet wird.
    \item[\hypertarget{CI}{CI}] Continuous Integration - Ein Entwicklungsansatz, bei dem Änderungen an einer Software regelmäßig in ein gemeinsames Repository integriert werden.
    \item[\hypertarget{CD}{CD}] Continuous Deployment - Eine Softwarebereitstellungsmethode, die das automatisierte Deployment von Codeänderungen ermöglicht.
    \item[\hypertarget{CSS}{CSS}] Cascading Style Sheets - Eine Stylesheet-Sprache, die verwendet wird, um das Layout von Webdokumenten zu gestalten.
    \item[\hypertarget{InlineCSS}{Inline CSS}] Eine Methode zur Anwendung von CSS-Stilen direkt innerhalb eines HTML-Elements, wodurch Stile spezifisch für dieses Element definiert werden.
    \item[\hypertarget{Stylesheets}{Stylesheets}] Dokumente, die CSS-Regeln enthalten und dazu verwendet werden, das Erscheinungsbild von HTML-Dokumenten zu steuern.
    \item[\hypertarget{Kubernetes}{Kubernetes}] Ein Open-Source-Container-Orchestrierungssystem, das zur Automatisierung von Bereitstellung, Skalierung und Verwaltung von containerisierten Anwendungen dient.
    \item[\hypertarget{YAML}{YAML}] A human-readable data serialization standard that is often used for configuration files and in applications where data is being stored or transmitted.
    \item[\hypertarget{ContainerImage}{Container Image}] Ein leichtgewichtiges, ausführbares Softwarepaket, das alles enthält, was benötigt wird, um eine Anwendung auszuführen, einschließlich Code, Laufzeit, Bibliotheken und Umgebungsvariablen.
    \item[\hypertarget{Ports}{Ports}] Virtuelle Punkte, die in der Computer-Netzwerk-Architektur verwendet werden, um spezifische Kommunikationskanäle zwischen Softwareanwendungen zu ermöglichen.
    \item[\hypertarget{Umgebungsvariablen}{Umgebungsvariablen}] Dynamische Werte, die von Prozessen verwendet werden, um Informationen über die Umgebung, in der sie ausgeführt werden, zu speichern und zu übergeben.
    \item[\hypertarget{KubernetesIngress}{Kubernetes Ingress}] Eine API-Objekt, das den Zugriff auf Services innerhalb eines Kubernetes-Clusters über HTTP und HTTPS regelt.
    \item[\hypertarget{Deployment}{Deployment}] Ein Kubernetes-Objekt, das sicherstellt, dass eine bestimmte Anzahl von Pods zu jedem Zeitpunkt ausgeführt wird und Updates von Anwendungen verwaltet.
    \item[\hypertarget{InfrastructureAsCode}{Infrastructure as Code}] Eine Praxis in der Softwareentwicklung, bei der die Infrastruktur über Code verwaltet und bereitgestellt wird, um Konsistenz und Wiederholbarkeit zu gewährleisten.
    \item[\hypertarget{Terraform}{Terraform}] Ein Open-Source-Werkzeug zur Verwaltung von Infrastruktur als Code, das es ermöglicht, Infrastruktur mithilfe von Konfigurationsdateien bereitzustellen, zu ändern und zu versionieren.
    \item[\hypertarget{ERM}{ERM}] Entity-Relationship-Modell - Ein konzeptionelles Datenmodell, das die Beziehungen zwischen Entitäten beschreibt.
    \item[\hypertarget{Go}{Go}] Programmiersprache Go - Eine von Google entwickelte Programmiersprache.
    \item[\hypertarget{HTML}{HTML}] Hypertext Markup Language - Die Standardauszeichnungssprache für Dokumente, die im Web angezeigt werden.
    \item[\hypertarget{GitLab}{GitLab}] Eine Plattform für Versionskontrolle und DevOps, die es Entwicklern ermöglicht, Code zu speichern, zu verwalten und zu teilen.
    \item[\hypertarget{OneSource}{OneSource}] Die GitLab-Instanz des Unternehmens, die als Code-Repository dient und die Zusammenarbeit an Softwareprojekten ermöglicht.
    \item[\hypertarget{Runway}{Runway}] Ein interner Dokumentationsort und eine Entwicklerplattform, die Ressourcen und Informationen für die Teammitglieder bereitstellt.
    \item[\hypertarget{GitLabSystemhooks}{GitLab-Systemhooks}] Mechanismen in GitLab, die es ermöglichen, auf bestimmte Ereignisse (wie Pushes, Merge-Requests usw.) zu reagieren und automatisierte Aktionen auszulösen, z.B. das Auslösen von CI/CD-Pipelines oder Benachrichtigungen in externen Systemen.
    \item[\hypertarget{GitLabEvent}{GitLab Event}] Ereignisse, die innerhalb der GitLab-Plattform auftreten, wie z.B. Push-Events, Merge-Requests oder Issues, auf die durch Systemhooks oder Integrationen reagiert werden kann.
    \item[\hypertarget{GitCommit}{Git Commit}] Der Vorgang, bei dem Änderungen an einem Repository gespeichert werden, um eine Version des Codes zu erstellen.
    \item[\hypertarget{Onboarding}{Onboarding}] Der Prozess, durch den neue Mitarbeiter in ein Unternehmen integriert werden und die notwendige Einarbeitung erhalten.
    \item[\hypertarget{UserExperience}{User Experience (UX)}] Die Gesamterfahrung eines Benutzers bei der Interaktion mit einem Produkt, insbesondere in Bezug auf Benutzerfreundlichkeit und Zufriedenheit.
    \item[\hypertarget{Schnittstelle}{Schnittstelle}] Ein definierter Punkt, an dem zwei Systeme oder Komponenten miteinander kommunizieren und Daten austauschen.
    \item[\hypertarget{Automatisierung}{Automatisierung}] Der Einsatz von Technologien, um Prozesse ohne menschliches Eingreifen auszuführen, um Effizienz und Genauigkeit zu erhöhen.
    \item[\hypertarget{Jira}{Jira}] Eine Projektmanagement-Software, die für die Planung und Nachverfolgung von Aufgaben in Softwareentwicklungsprojekten genutzt wird.
    \item[\hypertarget{agil}{agil}] Ein Entwicklungsansatz, der auf Flexibilität und Anpassungsfähigkeit abzielt und kurze Iterationen sowie regelmäßiges Feedback fördert.
    \item[\hypertarget{iterativ}{iterativ}] Ein Ansatz, bei dem ein Prozess in wiederholten Zyklen durchgeführt wird, um kontinuierliche Verbesserungen basierend auf Feedback zu ermöglichen.
    \item[\hypertarget{Microservices}{Microservices}] Ein Architekturmuster, bei dem eine Anwendung aus kleinen, unabhängigen Diensten besteht, die jeweils eine spezifische Funktionalität bereitstellen und über APIs kommunizieren.
    \item[\hypertarget{CodeReview}{Code Review}] Der Prozess, bei dem der Quellcode eines Entwicklers von anderen Entwicklern überprüft wird, um Fehler zu finden, den Code zu verbessern und Best Practices zu gewährleisten.
    \item[\hypertarget{UnitTests}{Unit Tests}] Tests, die einzelne Komponenten oder Module einer Software isoliert überprüfen, um sicherzustellen, dass sie wie erwartet funktionieren.
    \item[\hypertarget{IntegrationTests}{Integration Tests}] Tests, die das Zusammenspiel mehrerer Komponenten oder Systeme überprüfen, um sicherzustellen, dass sie gemeinsam korrekt arbeiten.
    \item[\hypertarget{BestPractices}{Best Practices}] Empfohlene Verfahren oder Techniken, die sich in der Praxis als effektiv und effizient erwiesen haben und die Qualität und Effizienz von Softwareentwicklungsprozessen erhöhen.
    \item[\hypertarget{SQS}{SQS}] Simple Queue Service - Ein verwalteter Nachrichtenwarteschlangenservice von AWS, der es Anwendungen ermöglicht, Nachrichten zwischen verschiedenen Komponenten auszutauschen und asynchrone Kommunikation zu unterstützen.
    \item[\hypertarget{SNS}{SNS}] Simple Notification Service - Ein verwalteter Nachrichten- und Benachrichtigungsdienst von AWS, der die Verteilung von Benachrichtigungen an mehrere Abonnenten über verschiedene Protokolle ermöglicht, einschließlich E-Mail, SMS und HTTP.
    \item[\hypertarget{AWS}{AWS}] Amazon Web Services - Eine umfassende und weit verbreitete Cloud-Computing-Plattform von Amazon, die eine Vielzahl von Diensten wie Computing, Storage, Datenbanken, maschinelles Lernen und vieles mehr anbietet.
    \item[\hypertarget{SES}{SES}] Simple Email Service - Ein verwalteter E-Mail-Dienst von AWS, der es ermöglicht, E-Mails einfach und kostengünstig zu senden und zu empfangen, insbesondere für Marketing- und Benachrichtigungszwecke.
    \item[\hypertarget{EKS}{EKS}] Elastic Kubernetes Service - Ein verwalteter Kubernetes-Dienst von AWS, der die Bereitstellung, Verwaltung und Skalierung von containerisierten Anwendungen auf Kubernetes vereinfacht.
    \item[\hypertarget{HTTP}{HTTP}] Hypertext Transfer Protocol - Ein Protokoll zur Übertragung von Daten über das Web, das die Kommunikation zwischen Clients (z.B. Webbrowser) und Servern regelt.
    \item[\hypertarget{HTTPPOST}{HTTP POST}] Eine HTTP-Anfragemethode, die verwendet wird, um Daten an einen Server zu senden, typischerweise um eine Ressource zu erstellen oder zu aktualisieren.
    \item[\hypertarget{FilterPolicies}{Filter Policies}] Richtlinien, die definieren, welche Nachrichten oder Ereignisse an Abonnenten weitergeleitet werden, basierend auf bestimmten Attributen oder Kriterien, um eine gezielte und effiziente Kommunikation zu ermöglichen.
    \item[\hypertarget{GraphClient}{Graph Client}] Ein API-Client, der verwendet wird, um mit der Microsoft Graph API zu interagieren und Daten zu Benutzern, Gruppen und anderen Ressourcen in Microsoft 365 abzurufen oder zu verwalten.
    \item[\hypertarget{Payload}{Payload}] Die Daten, die in einer Nachricht oder Anfrage gesendet werden, insbesondere in Bezug auf die Informationen, die über Webhooks oder APIs übertragen werden.
    \item[\hypertarget{MessageAttributes}{Message Attributes}] Zusätzliche Informationen oder Metadaten, die mit einer Nachricht gesendet werden, um sie zu beschreiben oder um zu ermöglichen, dass Filterrichtlinien auf sie angewendet werden können.
    \item[\hypertarget{SQSConsumer}{SQS Consumer}] Eine Komponente oder Anwendung, die Nachrichten aus einer AWS SQS-Warteschlange abruft und verarbeitet, um asynchrone Kommunikation und die Handhabung von Benutzerereignissen zu ermöglichen.
    \item[\hypertarget{Flag}{Flag}] Ein Bezeichner oder eine Kennzeichnung, die verwendet wird, um bestimmte Bedingungen, Zustände oder Optionen in Code zu markieren. Flags können dazu dienen, verschiedene Verhaltensweisen in einer Anwendung zu steuern oder den Zustand von Variablen oder Prozessen anzuzeigen.
    \item[\hypertarget{Endpoints}{Endpunkte}] Die definierten URLs oder URIs, über die Clients mit einem Server oder einer API kommunizieren können, um Daten zu senden oder zu empfangen.
\end{description}
