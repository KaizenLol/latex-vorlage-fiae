% !TEX root = Projektdokumentation.tex

% Hinweis: allgemein bekannte IT-Begriffe wie Datenbank oder Programmiersprache müssen nicht erläutert werden,
%          aber ggfs. Fachbegriffe aus der Domäne des Prüflings (z.B. Versicherung).

\begin{itemize}
    \item \hypertarget{MicrosoftTeams}{Microsoft Teams}: Cooles Kommunikations-Tool, das für Teamarbeit und Kommunikation verwendet wird.
    \item \hypertarget{CI}{CI}: Continuous Integration - Ein Entwicklungsansatz, bei dem Änderungen an einer Software regelmäßig in ein gemeinsames Repository integriert werden.
    \item \hypertarget{CD}{CD}: Continuous Deployment - Eine Softwarebereitstellungsmethode, die das automatisierte Deployment von Codeänderungen ermöglicht.
    \item \hypertarget{CSS}{CSS}: Cascading Style Sheets - Eine Stylesheet-Sprache, die verwendet wird, um das Layout von Webdokumenten zu gestalten.
    \item \hypertarget{ERM}{ERM}: Entity-Relationship-Modell - Ein konzeptionelles Datenmodell, das die Beziehungen zwischen Entitäten beschreibt.
    \item \hypertarget{Go}{Go}: Programmiersprache Go - Eine von Google entwickelte Programmiersprache.
    \item \hypertarget{HTML}{HTML}: Hypertext Markup Language - Die Standardauszeichnungssprache für Dokumente, die im Web angezeigt werden.
    \item \hypertarget{GitLab}{GitLab}: Eine Plattform für Versionskontrolle und DevOps, die es Entwicklern ermöglicht, Code zu speichern, zu verwalten und zu teilen.
    \item \hypertarget{GitLabSystemhooks}{GitLab-Systemhooks}: Mechanismen in GitLab, die es ermöglichen, auf bestimmte Ereignisse (wie Pushes, Merge-Requests usw.) zu reagieren und automatisierte Aktionen auszulösen, z.B. das Auslösen von CI/CD-Pipelines oder Benachrichtigungen in externen Systemen.
    \item \hypertarget{Onboarding}{Onboarding}: Der Prozess, durch den neue Mitarbeiter in ein Unternehmen integriert werden und die notwendige Einarbeitung erhalten.
    \item \hypertarget{UserExperience}{User Experience (UX)}: Die Gesamterfahrung eines Benutzers bei der Interaktion mit einem Produkt, insbesondere in Bezug auf Benutzerfreundlichkeit und Zufriedenheit.
    \item \hypertarget{Schnittstelle}{Schnittstelle}: Ein definierter Punkt, an dem zwei Systeme oder Komponenten miteinander kommunizieren und Daten austauschen.
    \item \hypertarget{Automatisierung}{Automatisierung}: Der Einsatz von Technologien, um Prozesse ohne menschliches Eingreifen auszuführen, um Effizienz und Genauigkeit zu erhöhen.
    \item \hypertarget{Jira}{Jira}: Eine Projektmanagement-Software, die für die Planung und Nachverfolgung von Aufgaben in Softwareentwicklungsprojekten genutzt wird.
    \item \hypertarget{agil}{agil}: Ein Entwicklungsansatz, der auf Flexibilität und Anpassungsfähigkeit abzielt und kurze Iterationen sowie regelmäßiges Feedback fördert.
    \item \hypertarget{iterativ}{iterativ}: Ein Ansatz, bei dem ein Prozess in wiederholten Zyklen durchgeführt wird, um kontinuierliche Verbesserungen basierend auf Feedback zu ermöglichen.
\end{itemize}
