% !TEX root = Projektdokumentation.tex

% Es werden nur die Abkürzungen aufgelistet, die mit \ac definiert und auch benutzt wurden. 
%
% \acro{VERSIS}{Versicherungsinformationssystem\acroextra{ (Bestandsführungssystem)}}
% Ergibt in der Liste: VERSIS Versicherungsinformationssystem (Bestandsführungssystem)
% Im Text aber: \ac{VERSIS} -> Versicherungsinformationssystem (VERSIS)

% Hinweis: allgemein bekannte Abkürzungen wie z.B. bzw. u.a. müssen nicht ins Abkürzungsverzeichnis aufgenommen werden.
% Hinweis: allgemein bekannte IT-Begriffe wie Datenbank oder Programmiersprache müssen nicht erläutert werden,
%          aber ggfs. Fachbegriffe aus der Domäne des Prüflings (z.B. Versicherung).

\begin{acronym}[WWWWW]
	\acro{API}{Application Programming Interface}
	\acro{CI}{Continuous Integration}
	\acro{CD}{Continuous Deployment}
	\acro{CSS}{Cascading Style Sheets}
	\acro{ERM}{Entity-Relationship-Modell}
	\acro{Go}{Programmiersprache Go}
	\acro{HTML}{Hypertext Markup Language}\acused{HTML}
	\acro{TUI}{Touristik Union International}
\end{acronym}

