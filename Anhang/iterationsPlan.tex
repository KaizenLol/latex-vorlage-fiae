\subsection{Iterationsplan}
\label{app:Iterationsplan}

Der folgende Iterationsplan beschreibt die Schritte der Implementierungsphase und deren Reihenfolge. Jede Funktionalität wurde zunächst auf der Testumgebung implementiert und nach erfolgreicher Validierung in die Produktionsumgebung auf dem bestehenden \textbf{Amazon EKS Cluster} ausgerollt:

\begin{enumerate}
    \item \textbf{Implementierung der GitLab System Hook:} Entwicklung der GitLab System Hook, die relevante GitLab-Ereignisse an ein SNS Topic sendet.
    
    \item \textbf{Erstellung des SNS Topics mit Terraform:} Definition und Bereitstellung des SNS Topics mittels Terraform, um die GitLab-Ereignisse zu empfangen.
    
    \item \textbf{Erstellung der SQS Queue mit Filter Policy:} Definition und Bereitstellung einer SQS Queue mit Terraform, inklusive einer Filter Policy, um nur \texttt{user\_create}-Ereignisse vom SNS Topic zuzulassen.
    
    \item \textbf{Implementierung des Dev-Kickstarter Services:} Entwicklung des in Go geschriebenen Dev-Kickstarter Services, der die \texttt{user\_create}-Ereignisse von der SQS Queue konsumiert.
    
    \item \textbf{Extraktion und Verarbeitung der relevanten Daten:} Extraktion der relevanten Informationen, wie die E-Mail-Adresse des erstellten Benutzers, aus den empfangenen GitLab-Ereignissen.
    
    \item \textbf{Versand von E-Mails und Hinzufügen zu MS Teams:} Implementierung der Logik, um mithilfe von AWS SES eine Willkommens-E-Mail an den neuen Benutzer zu senden und ihn zu einem MS Teams-Channel hinzuzufügen.
\end{enumerate}

Jeder dieser Schritte wurde iterativ auf der Testumgebung getestet und anschließend auf dem Amazon EKS Cluster in die Produktionsumgebung ausgerollt. Der vollständige Iterationsplan befindet sich im Anhang A.15: Iterationsplan auf S. xiii.
