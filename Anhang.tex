% !TEX root = Projektdokumentation.tex
\section{Anhang}
\subsection{Detaillierte Zeitplanung}
\label{app:Zeitplanung}

\tabelleAnhang{ZeitplanungKomplett}
\clearpage

\subsection{Ressourcen Planung}
\label{app:Ressourcen}
\textbf{Hardware}
\begin{itemize}
    \item Büroarbeitsplatz mit Thin-Client
\end{itemize}

\textbf{Software}
\begin{itemize}
    \item Windows 7 Enterprise mit Service Pack 1 - Betriebssystem
    \item Visual Studio Code - Hauptentwicklungsumgebung
    \item Docker - Containerisierung von Anwendungen
    \item Go - Programmiersprache zur Entwicklung des Webservers
    \item AWS - Cloud-Infrastruktur
    \item EKS (Elastic Kubernetes Service) - Orchestrierung der Container
    \item Terraform - Infrastruktur als Code
    \item GitLab - Versionskontrolle und CI/CD
    \item MS Teams - Zusammenarbeit im Team
    \item Jira - Projektmanagement
\end{itemize}

\textbf{Personal}
\begin{itemize}
    \item Entwickler - Umsetzung des Projektes
    \item Anwendungsentwickler - Review des Codes
\end{itemize}

\clearpage
\subsection{Iterationsplan}
\label{app:Iterationsplan}

Der folgende Iterationsplan beschreibt die Schritte der Implementierungsphase und deren Reihenfolge. Jede Funktionalität wurde zunächst auf der Testumgebung implementiert und nach erfolgreicher Validierung in die Produktionsumgebung auf dem bestehenden \textbf{Amazon EKS Cluster} ausgerollt:

\begin{enumerate}
    \item \textbf{Implementierung der GitLab System Hook:} Entwicklung der GitLab System Hook, die relevante GitLab-Ereignisse an ein SNS Topic sendet.
    
    \item \textbf{Erstellung des SNS Topics mit Terraform:} Definition und Bereitstellung des SNS Topics mittels Terraform, um die GitLab-Ereignisse zu empfangen.
    
    \item \textbf{Erstellung der SQS Queue mit Filter Policy:} Definition und Bereitstellung einer SQS Queue mit Terraform, inklusive einer Filter Policy, um nur \texttt{user\_create}-Ereignisse vom SNS Topic zuzulassen.
    
    \item \textbf{Implementierung des Dev-Kickstarter Services:} Entwicklung des in Go geschriebenen Dev-Kickstarter Services, der die \texttt{user\_create}-Ereignisse von der SQS Queue konsumiert.
    
    \item \textbf{Extraktion und Verarbeitung der relevanten Daten:} Extraktion der relevanten Informationen, wie die E-Mail-Adresse des erstellten Benutzers, aus den empfangenen GitLab-Ereignissen.
    
    \item \textbf{Versand von E-Mails und Hinzufügen zu MS Teams:} Implementierung der Logik, um mithilfe von AWS SES eine Willkommens-E-Mail an den neuen Benutzer zu senden und ihn zu einem MS Teams-Channel hinzuzufügen.
\end{enumerate}

Jeder dieser Schritte wurde iterativ auf der Testumgebung getestet und anschließend auf dem Amazon EKS Cluster in die Produktionsumgebung ausgerollt. Der vollständige Iterationsplan befindet sich im Anhang A.15: Iterationsplan auf S. xiii.

\clearpage
\subsection{Amortisierung}
\label{app:User}
\begin{figure}[htb]
    \centering
    \includegraphicsKeepAspectRatio{totalUser.png}{0.8}
    \caption{Anstieg der registrierten Benutzer in der GitLab-Instanz}
    \label{fig:totalUser}
    \end{figure}
    
Ein durchschnittlicher monatlicher Zuwachs von 150 Nutzern zeigt die wachsende Akzeptanz und Nutzung der Plattform. Dieser Trend verdeutlicht den steigenden Bedarf an GitLab als zentrale Lösung für die Projektverwaltung und -zusammenarbeit.
    
\clearpage
\subsection{Aktivitätsdiagramm für den Soll-Zustand}
\label{app:Akti}
\begin{figure}[htb]
\centering
\includegraphicsKeepAspectRatio{aktivity.drawio.png}{1.0}
\caption{Aktivitätsdiagramm für den Soll-Zustand}
\end{figure}

\clearpage
\subsection{Oberflächenentwürfe}
\label{app:Entwuerfe}
\begin{figure}[htb]
\centering
\includegraphicsKeepAspectRatio{mailSkizze.png}{0.7}
\caption{Erster Entwurf der E-Mail}
\end{figure}

\begin{figure}[htb]
\centering
\includegraphicsKeepAspectRatio{firstMail.png}{0.7}
\caption{Umsetzung in HTML mit CSS}
\end{figure}

\begin{figure}[htb]
\centering
\includegraphicsKeepAspectRatio{finalMail.png}{0.7}
\caption{Finales Design der E-Mail}
\end{figure}

\clearpage
\subsection{Screenshot der versendeten E-Mail}
\label{Screenshots}
\begin{figure}[htb]
    \centering
    \includegraphicsKeepAspectRatio{finalMail.png}{0.7}
    \caption{Finales Design der E-Mail}
    \label{app:finalDesign}
    \end{figure}


\clearpage

\subsection{Initialisierung der Hook-Struktur}
\label{app:Test}
\lstinputlisting[language=go, caption={Initialisierung der Hook-Struktur}]{Listings/hookStruktur.go}
\clearpage

\clearpage

\subsection{Verarbeitung eingehender Webhooks und Veröffentlichung von Ereignissen über AWS SNS}
\label{app:CNMI}
\lstinputlisting[language=php, caption={Verarbeitung eingehender Webhooks und Veröffentlichung von Ereignissen über AWS SNS}]{Listings/cnmi.php}
\clearpage

\subsection{Initialisierung der Kickstarter-Struktur}
\label{app:kickStruct}
\lstinputlisting[language=php, caption={Initialisierung der Kickstarter-Struktur}]{Listings/kickstarterStruct.go}
\clearpage

\subsection{Verarbeitung des Gitlab Events und senden der E-Mail/Hinzufügen zur Teams Gruppe}
\label{app:kickMain}
\lstinputlisting[language=php, caption={Initialisierung der Kickstarter-Struktur}]{Listings/kickMain.go}
\clearpage

\subsection{Deployment yaml für die Systemhook}
\label{app:yamlFile}
\lstinputlisting[language=php, caption={Deployment yaml für die Systemhook}]{Listings/yamlHook.yml}
\clearpage

\subsection{Terraform Infrastructure as Code}
\label{app:terraform}
\lstinputlisting[language=php, caption={Terraform Infrastructure as Code}]{Listings/terraform.tf}
\clearpage

\subsection{Entwicklerdokumentation (Auszug)}
\label{app:Doc}
\begin{figure}[htb]
\centering
\includegraphicsKeepAspectRatio{enwicklerdoc.png}{1.1}
\caption{Entwicklerdokumentation}
\end{figure}

    
\clearpage